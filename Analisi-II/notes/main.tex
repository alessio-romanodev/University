\documentclass{article}
\input{header.tex}
\title{Analisi II\\ \large\textbf{Barbato Rosa} \\ a.a. 2024-2025}

\author{\textbf{Author}\\ Alessio Romano}

\begin{document}
\maketitle

\newpage
\tableofcontents
\newpage

\section{Successioni  e Serie di Funzioni}
\subsection{Successioni di funzioni}
\begin{definition}[Successione di funzioni]
  Sia $X \subseteq \mathbb{R}$ un sottoinsieme di $\mathbb{R}$. Una \textit{successione di funzioni} è un'applicazione matematica tale che:
  \[\{f_n\}_{n\in \NaturalNumbers}:n \in \mathbb{N} \to F_n(x)\]
  Dove $f_n(x): X \to \mathbb{R}$
\end{definition}
Prendiamo per esempio la successione di funzioni $f_n(x) = x^n$, potremmo chiederci intuitivamente che cosa succede alla successione di funzioni quando $n \to +\infty$. Per farlo fissiamo $x$, Si ottiene che
\[\lim_{n\to\infty} x^n = \begin{cases}
    0 \text{ se } -1<x<1\\
    1 \text{ se } x=1\\
    +\infty \text{ se } x>1\\
    \nexists \text{ se } x \leq -1
\end{cases}\]
Abbiamo introdotto intuitivamente il concetto di \textbf{convergenza puntuale}. Definiamo ora la convergenza puntuale rigorosamente
\begin{definition}[Convergenza puntuale]
  Data una successione di funzioni $f_n(x)$ tale che $X \to \Real$, tale successione \textit{converge puntualmente} ad $f(x)$, in $X$ se e solo se
  \[
    \lim_{n\to\infty} f_n(x) = f(x), \;\;\; \forall x \in X
  \] 
  Ossia
  \[
    \forall x \in X, \forall \epsilon >0, \exists \nu = \nu_{\epsilon, x}: \forall n>\nu, |f_n(x)-f(x)|<\epsilon 
  \]
\end{definition}
Insomma, stiamo dicendo che data una successione di funzioni, se per ogni $x \in X,  \epsilon > 0$ esiste un certo indice della successione $\nu$ dipendente dal punto $x$ e il valore $\epsilon$, tale che preso un valore $n$ della successione più grande dell'indice $nu$, la distanza tra la funzione limite $f(x)$ e la successione di  funzioni è sempre minore di $\epsilon$\dots
\begin{definition}[Convergenza Uniforme]
  $f_n$ \textit{converge uniformemente} a $f(x)$ in $x$ se e solo se 
  \[\forall \epsilon>0, \exists \nu = \nu_\epsilon: \forall n> \nu, |f_n(x)-f(x)|<\epsilon, \;\;\; \forall x \in X\]
\end{definition}
Dunque per convergenza uniforme si intende che dopo un certo indice $\nu$, dipendente unicamente da $\epsilon$, la distanza tra la successione di funzioni e la funzione limite è minore di $\epsilon$, non più in un unico punto $x_0$, ma su tutto l'intervallo. Si noti che l'uniforme convergenza implica la convergenza puntuale, ma l'inverso non è sempre valido
\begin{theorem}[Criterio di Cauchy convergenza uniforme]
  $f_n(x)\to f$ uniformemente in $X$ se e solo se
    \[\forall \epsilon > 0, \exists \nu = \nu_\epsilon: \sup_{x\in X}|f_n(x)-f(x)|\leq \epsilon\]
\end{theorem}
In altre parole, una successione di funzioni converge uniformemente se e solo se il limite per $n\to\infty$ del suo estremo superiore tende a $0$
\begin{theorem}[Continuità del limite]
  Sia $\{f_n\}$ una successione di funzioni. Se $f_n \to f$ uniformemente in $X$ e $f_n$ è continua $\Rightarrow f$ è continua in $X$ 
\end{theorem}
\begin{theorem}[Passaggio al limite sotto il segno di derivata]
Sia $f_n$ una successione di funzione continua in [a,b], con $f_n \to f$ uniformemente in $[a,b]$ allora
\[\lim_{n\to\infty} \int_{a}^{b} f_n(x)dx = \int_{a}^{b} \lim_{n\to\infty} f_n(x)dx\]  
\end{theorem}
\begin{theorem}[Criterio di Cauchy uniforme]
  $f_n \to f$ uniformemente in $X$ se e solo se 
  \[\forall \epsilon > 0, \exists \nu = \nu_\epsilon: |f_n(x)-f_m(x)<\epsilon, \;\;\; \forall m,n > \nu_\epsilon, \forall x\in X\]
\end{theorem}
\begin{theorem}
  Sia $\{f_n\}$ una successione di funzioni (derivabile con derivate continue). Se
  \begin{itemize}
    \item $\exists x_0 \in [a,b]: f_n (x_0)$ converge
    \item $f'n$ converge uniformemente in $[a,b]$ ad una funzione $g$
  \end{itemize}
  Allora $f_n$ converge uniformemente ad una funzione $f$ con derivata continua in $[a,b]$ e 
  \[\lim_n f_n'(x)=f'(x)=(\lim_n f_n (x))'\]
\end{theorem}
\subsection{Serie di funzioni}
\begin{definition}[Serie di funzioni]
  Una serie di funzioni è un oggetto della forma
  \[\sum_{n=1}^{\infty} f_n(x)\]
  Ossia una somma infinita di funzioni $f_n: A\to\Real$ definite sull'insieme $A$. Data la successione di \textbf{somme parziali}, $s_1,s_2,...,s_n$:
  \[s_1(x)=f_1(x)\]
  \[s_2(x)=f_1(x)+f_2(x)\]
  \[...\]
  \[s_n(x)=f_1(x)+f_2(x)+...+f_n(x)\]
  Si definisce $s_n$ come la \textbf{Serie di funzioni} associata alla successione $f_n(x)$. Se questa serie converge, convergerà ad una certa funzione $F(x)$, dove:
  \[\sum_{n=1}^{+\infty}f_n(x)= F(x)\]
\end{definition}
Procediamo definendo i due tipi di convergenza a cui una serie di funzioni può essere soggetta:
\begin{definition}[Convergenza Puntuale]
  Data una serie di funzioni $\sum_{n=1}^{+\infty}f_n(x)$ Fissato un valore numerico per $x = x_0 \in \Real$, si viene a formare una serie numerica
  \[\sum_{n=1}^{+\infty}f_n(x_0)\]
\end{definition}
Se converge, la serie di partenza converge in $x_0$, l'insieme dei valori di $x$ che rendono la serie convergente è detto \textbf{Insieme di convergenza}
\begin{definition}
  Consideriamo una serie di funzioni 
  \[\sum_{n=1}^{+\infty}f_n(x)\]
  Quest'ultima converge uniformemente su un insieme $A$, se le sue somme parziali, convergono uniformemente alla funzione limite $F(x)$, ossia:
  \[\forall \epsilon>0, \exists \nu \in \mathbb{N}: \forall n>\nu, \forall x \in A |s_n(x)-F(x)|>\epsilon\]
\end{definition}














\end{document}
