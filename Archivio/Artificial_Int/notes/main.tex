\documentclass[12pt, a4paper]{article}

\input{header.tex}

\title{Intelligenza Artificiale\\ \large\textbf{Sperli' Giancarlo} \\ a.a. 2023-2024}

\author{\textbf{Author}\\ Alessio Romano}

\begin{document}
\maketitle

\newpage
\tableofcontents
\newpage

\section{Agenti e Ambienti}
Definiamo \textbf{agente} come qualsiasi cosa che possa essere vista come un sistema che percepisce l'ambiente in cui opera mediante sensori e agisce su di esso mediante attuatori. L'\textbf{ambiente} può essere qualsiasi cosa ma nelle pratica è quella parte che influenza ciò che l'agente percepisce e sulla quale influiscono le azioni dell'agente. 
\begin{itemize}
  \item Per effettuare la descrizione di un agente, è necessario specificare per ogni sequenza percettiva, l'azione intrapresa dall'agente.
\end{itemize}
\subsection{Agenti razionali}
Un \textbf{agente razionale}, è un agente che \textit{fa la cosa giusta}. In ambito di intelligenza artificiale, ci si riferisce alla nozione di \textbf{consequenzialismo}, dunque valutiamo il comportamento di un agente considerando le consequenze delle sequenze di azioni intraprese da quest ultimo. La razionalità dipende da 4 fattori principali:
\begin{itemize}
  \item La misura di prestazione che definisce il criterio del successo
  \item La conoscenza pregressa dell'agente
  \item Le azioni che l'agente può effettuare
  \item La sequenza percettiva dell'agente fino all'istante corrente
\end{itemize}
Riformulando possiamo definire un \textbf{agente razionale}, come un agente che per ogni possibile sequenza di percezioni, sceglie azioni che massimizzano il valore atteso della sua misura di prestazione. Dunque un agente razionale non può prescindere da:
\begin{itemize}
  \item \textbf{Misura di prestazione}
  \item \textbf{Conoscenza dell'ambiente}
  \item \textbf{Sensori}
  \item \textbf{Attuatori}
\end{itemize}
\subsection{Ambienti}
Il concetto i ambiente si riferisce a tutto ciò con cui un agente interagisce. Un ambiente ha diverse proprietà:
\begin{itemize}
  \item \textbf{Osservabilità}: 
    \begin{itemize}
      \item se i sensori di un agente gli danno accesso allo stato completo dell'ambiente in ogni momento allora diciamo che l'ambiente operativo è \textbf{completamente osservabile} 
      \item se solo parte dell'ambiente è percettibile in un dato momento allora si dice \textbf{parzialmente osservabile}
      \item se l'agente non è in grado di osservare l'ambiente si dice \textbf{inosservabile}.
    \end{itemize}
  \item \textbf{Singolarità dell'agente}: 
    \begin{itemize}
      \item Un agente che opera da solo in un ambiente definisce un \textbf{ambiente singolo}
      \item Un agente che opera con altri agenti in un ambiente definisce un \textbf{ambiente multi-agente}
    \end{itemize}
  \item \textbf{Deterministicità}:
    \begin{itemize}
      \item Se lo stato successivo dell'ambiente è completamente determinato dallo stato corrente dell´ambiente e dall'azione eseguita dall´agente, si parla di \textbf{ambiente deterministico}
      \item Se lo stato successivo non dipende unicamente dallo stato corrente dell'ambiente e dall'azione eseguita dall'agente si parla di \textbf{ambiente non deterministico}
    \end{itemize}
  \item \textbf{Episodicità/Sequenzialità}:
    \begin{itemize}
      \item Un ambiente si definisce \textbf{episodico}, se ogni scelta dell'agente non dipende dalla sequenza di azioni intraprese in precedenza
      \item Un ambiente si dice \textbf{sequenziale} se ogni scelta dell'agente dipende dalle scelte precedenti
    \end{itemize}
  \item \textbf{Staticità/Dinamicità}:
    \begin{itemize}
      \item Un ambiente si dice \textbf{statico} se non cambia mentre un agente decide come agire
      \item Un ambiente si dice \textbf{dinamico} se varia mentre un agente decide come agire
    \end{itemize}
  \item \textbf{Continuità/Discretezza}: La distinzione tra discreto e continuo si applica allo stato dell'ambiente, al modo in cui è gestito il tempo, alle percezioni e azioni dell'agente
  \item \textbf{Noto/Ignoto}: questa distinzione non si riferisce all'ambiente in se, ma allo stato di conoscenza dell'ambiente delle leggi fisiche dell'ambiente stesso. In un ambiente noto, sono conosciuti i risultati per tutte le azioni, in uno ignoto l'agente dovrà apprendere come funziona per poter prendere buone decisioni
\end{itemize}
Il compito dell'itelligenza artificiale è progettare il programma agnte che implementa la funzione agente, e dunque fa corrispondere le percezioni alle azioni.
\subsection{Struttura degli agenti}
Abbiamo 4 tipi di strutture principali per il programma agente:
\begin{enumerate}
  \item \textbf{Agenti reattivi semplici}: Agenti che si occupano di scegliere le azioni sulla base della percezione corrente, ignorando la storia percettiva precedente.
  \item \textbf{Agenti reattivi basati su modello}: Agenti che mantengono uno stato interno che dipende dalla storia delle percezioni e che quindi riflette almeno una parte degli aspetti non osservabili dello stato corrente. Aggiornare l'informazione dello stato interno al passaggio del tempo richiede a questo tipo di agenti due tipi di conoscenza
    \begin{itemize}
      \item Conoscenza sull'evoluzione del mondo nel tempo, suddivisa in effetti delle azioni dell'agente e modalità di evoluzione del mondo indipendente dall'agente (chiamata \textbf{modello di transizione del mondo})
      \item Conoscenza su come lo stato del mondo si rifletta nelle percezioni dell'agente (chiamata \textbf{conoscenza sensoriale})
    \end{itemize}
  Un agente che utilizza tali modelli si definisce agente basato su modello
  \item \textbf{Agenti basati su obiettivi}: oltre che della descrizione dello stato corrente, l'agente ha bisogno di informazioni riguardanti il suo obiettivo. Il programma agente può unire quest'informazione al modello per scegliere azioni che portino all'obiettivo
  \item \textbf{Agenti basati su utilità}: nella maggior parte degli ambienti, i soli obiettivi non sono sufficienti a generare un comportamento di alta qualità. Ci serve dunque una misura di prestazione generale per poter confrontare stati del mondo differenti e misurare precisamente l'utilità che porterebbero all'agente. Una funzione di utilità è in sostanza, l'internalizzazione della misura di prestazione da parte dell'agente. Un agente razionale basato sull'utilità, sceglie l'azione che massimizza quest'ultima.
\end{enumerate}
\section{Risoluzione dei problemi}
Quando l'azione giusta non è subito evidente, un agente può avere la necessità di \textit{guardare in avanti}, ossia considerare una sequenza di azioni che formano un cammino che porterà a uno stato obiettivo. Questo tipo di agente è definito \textbf{agente risolutore di problemi}, e il processo computazionale che effettua è la ricerca. Gli agenti risolutori utilizzano rappresentazioni \textbf{atomiche} in cui gli stati del mondo sono considerati come entità prive di una struttura interna visibile agli algoritmi per la risoluzione dei problemi. 
\subsection{Agenti risolutori di problemi}
Se l'ambiente è ignoto, l'agente può eseguire esclusivamente azioni scelte a caso. Supponiamo un ambiente in cui l'agente può sempre accedere a informazioni sul mondo. Il processo di risoluzione del problema si compone di 4 fasi.
\begin{itemize}
  \item \textbf{Formulazione dell'obiettivo}: l'agente adotta l'obiettivo di raggiungere l'ambiente.
  \item \textbf{Formulazione del problema}: l'agente elabora una descrizione degli stati e delle lezioni del mondo interessato.
  \item \textbf{Ricerca}: prima di effettuare qualsiasi azione nel mondo reale, l'agente simula nel suo modello sequenze di azioni, continuando a cercare finchè trova una sequenza che raggiunge l'obiettivo, denominata \textbf{soluzione}, oppure troverà che non è possibile alcuna soluzione.
  \item \textbf{Esecuzione}: l'agente ora può eseguire le azioni specificate nella soluzione, una per volta
\end{itemize}
Si noti che uan volta che l'agente ha trovato una soluzione, può ignorare le proprie percezioni e eseguire semplicemente le azioni. Nella teoria del controllo si parla in questo caso di sistema ad \textbf{anello aperto}, perchè ignorando le percezioni si rompe il ciclo tra agente e ambiente. Nel caso che l'ambiente non sia deterministico, o il modello errato, l'agente sarebbe più sicuro usando un approccio ad \textbf{anello chiuso}, che monitora le percezioni
\subsection{Problemi di ricerca}
Definiamo un problema di ricerca come:
\begin{itemize}
  \item l'insieme di possibili stati in cui può trovarsi l'ambiente, chiamato \textbf{spazio degli stati}
  \item lo \textbf{stato iniziale} in cui si trova l'agente inizialmente
  \item un insieme di uno o più \textbf{stati obiettivo}
  \item le \textbf{azioni possibili} dell'agente
  \item un \textbf{modello di transizione}, che descrive ciò che fa ogni azione
  \item una \textbf{funzione costo dell'azione} che restituisce il costo numerico di applicare l'azione $a$ nello stato $s$ per raggiungere uno stato $s'$
\end{itemize}
Una sequenza di azioni si definisce un \textbf{cammino}. Una soluzione è un cammino che porta dallo stato iniziale allo stato obiettivo e può essere:
\begin{itemize}
  \item \textbf{Soluzione ottimale}: se tra tutte le possibili soluzioni ha il costo minimo
  \item \textbf{Soluzione subottimale}: si limita a trovare una soluzione, ma non minimizza il costo
\end{itemize}
\section{Algoritmi di ricerca}
Un algoritmo di ricerca riceve in input un problema di ricerca e resituisce una soluzione o un'indicazione di fallimento. Consideriamo algoritmi che sovrappongono un \textbf{albero di ricerca} al grafo dello spazio degli stati, fornendo vari cammini a partire dallo stato iniziale e cercandone uno che raggiunga uno stato obiettivo. Ciascun \textbf{nodo} nell'albero di ricerca corrisponde ad uno stato nello spazio degli stati e i rami dell'albero corrispondono ad azioni.
\subsection{Strutture dati per la ricerca}
Gli algoritmi di ricerca chiedono una struttura dati per tenere traccia dell'albero di ricerca. Un \textbf{nodo} nell'albero è rappresentato da una struttura dati con 4/5 componenti:
\begin{itemize}
  \item \textbf{n.STATO}: lo stato in cui corrisponde il nodo
  \item \textbf{n.PADRE}: il nodo dell'albero di ricerca che ha generato il nodo corrente
  \item \textbf{n.AZIONE}: l'azione applicata allo stato padre per generare il nodo corrente
  \item \textbf{n.COSTO\_CAMMINO}: il costo totale per arrivare dallo stato iniziale al nodo corrente
  \item \textbf{n.EURISTICA}: una funzione euristica $h(n)$ se utilizzata (caso ricerca informata)
\end{itemize}
Seguendo i puntatori padre si risale agli stati e alle azioni lungo il cammino fino a tale nodo. Ci serve ora una struttura dati per memorizzare la \textbf{frontiera}. Utilizziamo una \textbf{coda} con le seguenti operazioni
\begin{itemize}
  \item \textbf{VUOTA?(frontiera)}: restituisce true se non ci sono più nodi nella frontiera
  \item \textbf{POP(frontiera)}: rimuove il nodo in cima alla frontiera e lo restituisce
  \item \textbf{TOP(frontiera)}: restit
\end{itemize}











\end{document}

